%regresní analýza
\chapter{Regresní analýza}
\section{Jednorozměrná lineární regrese}
%příklad matlab
Předpokládejme, že se~sledují dvě fyzikální veličiny $X$ a~$Y$ mezi~kterými existuje lineární závislost
$$Y=\beta_0+\beta_1X .$$
$\beta_0$ a~$\beta_1$ nejsou známy, a~proto se~provádí experiment, při~němž se~zjišťují hodnoty dvojic $(X,Y)$. Často se~stává, že měření hodnot $X$ probíhá prakticky zcela přesně (například $X$ se~nastavuje na~předem dané úrovně), zatímco $Y$ se~měří s~určitou chybou. Zavádí se~tedy model
$$Y_i=\beta_0 +\beta_1X_i + e_i \quad \forall~i=1,...,n,$$
kde $e_i$ je náhodný šum a~$e_1,...,e_n$ jsou $iid~\NN(0,{\sigma}^2)$ a~dvojice $(x_1,y_1),...,(x_n,y_n)$ získáme měřením. Neznáme parametry jsou $\beta_0,\beta_1, {\sigma}^2$, chtěli bychom je odhadnout na~základě výběru (MLE odhady).\\

Rozdělení $Y_i$ je $Y_i\sim\NN(\beta_0 + \beta_1x,{\sigma}^2)$, a~tedy věrohodnostní funkce výběru $y_1,...,y_n$ je
$$L=\Br{\frac{1}{\sqrt{2\pi{\sigma}^2}}}^n \e{-\frac{1}{2{\sigma}^2}\sum_{i=1}^{n}(Y_i-\beta_0-\beta_1x)^2}.$$
$$l=\ln L=-\frac{n}{2}\ln{2\pi}-\frac{n}{2}\ln{{\sigma}^2}-\frac{1}{2{\sigma}^2}\sum_{i=1}^{n}(Y_i-\beta_0-\beta_1x_i)^2.$$
Je zřejmé, že pro~libovolné ${\sigma}^2$ potřebujeme minimalizovat
$$S(\beta_0,\beta_1)=\sum_{i=1}^{n}(Y_i-\beta_0-\beta_1x_i)^2   $$
přes $\beta_0,\beta_1$, na~což použijeme metodu nejmenších čtverců (poznámka?).
$$\frac{\partial l}{\partial \beta_0}=2\frac{1}{2{\sigma}^2}\sum_{i=1}^{n}(Y_i-\beta_0-\beta_1x_i)=0,  $$
$$\frac{\partial l}{\partial \beta_1}=\frac{1}{{\sigma}^2}\sum_{i=1}^{n}(Y_i-\beta_0-\beta_1x_i) x_i=0. $$
Z toho pak
$$  \sum_{i=1}^{n} Y_i -n\beta_0-\beta_1 \sum_{i=1}^{n} x_i=0, $$
$$ \beta_0=\Oyn -\beta_1\oxnn=\frac{1}{n}\sum_{i=1}^{n}Y_i-\beta_1\frac{1}{n}\sum_{i=1}^{n}x_i.$$
Po vynásobení poslední rovnice $n$ úpravou dostaneme vztah
$$\sum_{i=1}^{n}(Y_i-\Oyn+\beta_1\oxnn-\beta_1x_i)x_i=0  $$
a následně i~vztah
$$ \sum_{i=1}^{n}Y_ix_i-\Oyn\sum_{i=1}^{n}+\beta_1\oxnn\sum_{i=1}^{n}x_i-\beta_1\sum_{i=1}^{n}x_i^2=0. $$
Z toho už následně vyjádříme
$$ \widehat{\beta}_1=\frac{\sum_{i=1}^{n}x_iY_i-n\Oyn\oxnn}{\sum_{i=1}^{n}x_i^2-n\oxnn^2}\qquad\text{a}\qquad \widehat{\beta}_0=\Oyn-\widehat{\beta}_1\oxnn. $$
Nyní již spočítáme logaritmickou věrohodnostní funkci
$$\frac{\partial l}{\partial ({\sigma}^2)}=-\frac{n}{2}\frac{1}{{\sigma}^2}+\frac{1}{2({\sigma}^2)^2}\sum_{i=1}^{n}(Y_i-\beta_0-\beta_1x_i)^2=0,$$
odkud
$$\hsn=\frac{1}{n}\sum_{i=1}^{n}(Y_i-\widehat{\beta}_0-\widehat{\beta}_1x_i)^2. $$
Pokud dále označíme
$$\widehat{Y}_i=\widehat{\beta}_0+\widehat{\beta}_1x_i, $$
pak rozdíly
$$r_i=Y_i-\widehat{Y}_i $$
nazýváme \textbf{rezidua} (která by měla mít normální rozdělení, aby byly splněny předpoklady modelu) a~
$$ \sum_{i=1}^{n}r_i^2=\sum_{i=1}^{n}(Y_i-\widehat{\beta}_0-\widehat{\beta}_1x_i)^2=S_e  $$
nazveme \textbf{reziduální součet čtverců}.
\subsection*{$R^2$ statistika}
Tuto statistiku definujeme vztahem
$$R^2=1-\frac{\sum_{i=1}^{n}r_i^2}{\sum_{i=1}^{n}(Y_i-\oyn)^2}=1-\frac{\sum_{i=1}^{n}(Y_i-\widehat{Y}_i)^2}{\sum_{i=1}^{n}(Y_i-\oyn)^2}, $$
který se~dá chápat jako podíl součtu reziduálních čtverců a~rozptylu $Y$.
$R^2$ se~interpretuje jako poměr variability v~datech vysvětlené lineárním modelem. Čím větší je $R^2$, tím lépe vysvětluje náš model data, v~ideálním případě pak $R^2=1$.
Dále bychom chtěli:
\begin{enumerate}
\item sestrojit IS pro~parametry modelu $\beta_0, \beta_1, {\sigma}^2$,

\item intervaly pro~predikci hodnoty $y$ v~daném bodě $x$ a

\item testovat hypotézy na~parametrech modelu, například F-stat. v~MATLABu testuje $H_0:\beta_0=0$ a~$\beta_1=0$, že vysvětlující proměnná $y$ není korelovaná s~vysvětlovanou proměnnou $x$.
\end{enumerate}


Vše je podobné testům o~parametrech $N(\mu,{\sigma}^2)$ (t-test, F-test), potřebujeme rozdělení odhadů $\widehat{\beta}_0,\widehat{\beta}_1, \widehat{{\sigma}^2}$. Sdružené rozdělení $\widehat{\beta}_0,\widehat{\beta}_1$ se~najde snadno, protože to jsou lineární funkce $Y_i$ takže budou mít normální rozdělení, stačí tedy určit střední hodnoty, rozptyly, kovariance,...\\
Označme výběrový rozptyl $x$ jako
$${\sigma_x}^2=\frac{1}{n}\sum_{i=1}^{n}x_i^2-\oxnn^2. $$
Platí, že
\begin{enumerate}
\item $$ \widehat{\beta}_1\sim \n{\beta_1,\frac{{\sigma}^2}{n{\sigma_x}^2}}, $$
$$\widehat{\beta}_0\sim \n{\beta_0,{\sigma}^2\Br{\frac{1}{n}+\frac{(\oxnn)^2}{n{\sigma_x}^2}}}=\n{\beta_0,\frac{{\sigma}^2}{n{\sigma_x}^2}\frac{1}{n}\sum_{i=1}^{n}x_i^2}, $$
$$\Cov(\widehat{\beta}_0,\widehat{\beta}_1)=-\frac{\oxnn{\sigma}^2}{n{\sigma_x}^2}, $$
\item $\widehat{\sigma}^2 $ je nezávislé na~$\widehat{\beta}_0$ a~$\widehat{\beta}_1$,
\item $$\frac{n\widehat{\sigma}^2}{\sigma^2} \sim {\chi}^2(n-2).$$
\end{enumerate}
%

\begin{remark}
První bod znamená, že $(\beta_0,\beta_1)\sim \NN(\mu,\sum)$, kde
$$\bmu=(\beta_0,\beta_1)\quad \text{a}\quad \sum=\frac{\sigma^2}{n\sigma_x^2}\begin{pmatrix}
\oxnn^2 & -\oxnn\\
-\oxnn & 1
\end{pmatrix}. $$
%\mu vektorově
\end{remark}


Konfidenční intervaly\\
\begin{enumerate}
\item $\sigma^2$, a~protože $\frac{n\widehat{\sigma}^2}{\sigma^2}\sim {\chi}^2(n-2)$, víme, že s~pravděpodobností $\PP=1-\alpha$ bude
$${\chi}^2_{\frac{\alpha}{2}}(n-2) \leq \frac{n\widehat{\sigma}^2}{\sigma^2} \leq {\chi}^2_{1-\frac{\alpha}{2}}(n-2),$$
a tedy $(1-\alpha)$\% IS (interval spolehlivosti) pro~$\sigma^2$ je
$$\frac{n\widehat{\sigma}^2}{{\chi}^2_{1-\frac{\alpha}{2}}(n-2)} \leq \sigma^2 \leq \frac{n\widehat{\sigma}^2}{{\chi}^2_{\frac{\alpha}{2}}(n-2)}.$$
\item $\beta_1$\\
Veličiny 
$\frac{\widehat{\beta}_1-\beta_1}{\sqrt{\frac{\sigma^2}{n\sigma_x^2}}}\sim \NN(0,1) $
a
$\frac{n\widehat{\sigma}^2}{\sigma^2} \sim {\chi}^2(n-2) $
jsou nezávislé. Z~toho vyplývá, že 
$$
 \frac{(\widehat{\beta}_1-\beta_1)\Big/\sqrt{\frac{\sigma^2}{n\sigma_x^2}}}{\sqrt{\frac{n\widehat{\sigma}^2}{\sigma^2}\frac{1}{n-2}}} \sim t(n-2).$$

Z toho potom
\begin{align}
 \frac{\widehat{\beta}_1-\beta_1}{\sqrt{\frac{\widehat{\sigma}^2}{(n-2)\sigma_x^2}}}=(\widehat{\beta}_1-\beta_1)\sqrt{\frac{(n-2)\sigma_x^2}{\widehat{\sigma}^2}}\sim t(n-2), \label{eq1}
\end{align}
což znamená, že
$$-t_{1-\frac{\alpha}{2}}(n-2) \leq (\widehat{\beta}_1-\beta_1)\sqrt{\frac{(n-2)\sigma_x^2}{\widehat{\sigma}^2}}   \leq t_{1-\frac{\alpha}{2}}(n-2)$$
s pravděpodobností $\PP=1-\alpha$, a~tedy
$$\widehat{\beta}_1-t_{1-\frac{\alpha}{2}}(n-2)\sqrt{\frac{\widehat{\sigma}^2}{(n-2)\sigma_x^2}}  \leq \beta_1 \leq \widehat{\beta}_1+t_{1-\frac{\alpha}{2}}(n-2)\sqrt{\frac{\widehat{\sigma}^2}{(n-2)\sigma_x^2}}$$
je $100(1-\alpha)$\% IS pro~$\beta_1$.
Podobně pro~$\beta_0$ dostaneme, že 
$$ \frac{\widehat{\beta}_0-\beta_0}{\sqrt{\sigma^2(\frac{1}{n}+\frac{\overline{x}_n^2}{\sigma_x^2})}} \frac{1}{\sqrt{\frac{n\widehat{\sigma}^2}{\sigma^2}\frac{1}{n-2}}} \sim t(n-2), $$
\begin{align}
\frac{\widehat{\beta}_0-\beta_0}{\sqrt{(1+\frac{\overline{x}_n^2}{\sigma_x^2})\widehat{\sigma}^2\frac{1}{n-2}}} \sim t(n-2), \label{eq2}
\end{align}
a tedy
$$\widehat{\beta}_0-t_{1-\frac{\alpha}{2}}(n-2){\sqrt{(1+\frac{\oxnn^2}{\sigma_x^2})\widehat{\sigma}^2\frac{1}{n-2}}} \leq \beta_0 \leq \widehat{\beta}_0+t_{1-\frac{\alpha}{2}}(n-2){\sqrt{(1+\frac{\oxnn^2}{\sigma_x^2})\widehat{\sigma}^2\frac{1}{n-2}}}$$
je $100(1-\alpha)$\% IS pro~$\beta_0$.
\end{enumerate}

Statistiky \eqref{eq1} a~\eqref{eq2} se~dají použít i~pro~konstrukci testů například $H_0: ~\beta_1=0$. Za~platnosti $H_0$ totiž
$$T_1=\widehat{\beta}_1\sqrt{\frac{(n-2)\sigma_x^2}{\widehat{\sigma2}}} \sim t(n-2),$$
a tedy $H_0$ zamítáme, pokud
$$|T_1|>t_{1-\frac{\alpha}{2}}(n-2).$$
$$ \test{|T_1|>t_{1-\frac{\alpha}{2}}(n-2)} $$
\begin{example}[Měření rychlosti zvuku v~závislosti na~teplotě]~\\
	
\begin{center}
\begin{tabular}{|l|l|l|l|l|l|}
\hline
\multicolumn{1}{|c|}{{ \textbf{teplota}}} & -20 & \multicolumn{1}{c|}{0} & 20  & 50  & 100 \\ \hline
\textbf{rychlost (m/s)}                                       & 323 & 327                    & 340 & 364 & 386 \\ \hline
\end{tabular}

\end{center}
\begin{align*}
&\Oxn=\frac{1}{n}\sum_{i=1}^{n}X_i=30,\quad\Oyn=348,\quad
\sum_{i=1}^{n}X_iY_i=57 140,\quad
\sum_{i=1}^{n}X_i^2=13 300,\\
&\sigma_x^2=\frac{1}{n}\sum_{i=1}^{n}X_i^2-\overline{X_n^2}=\frac{1}{5}13 300-900=1 760,\\
&\widehat{\beta}_1=\frac{\sum_{i=1}^{n}X_iY_i-5\oxnn\Oyn}{\sum_{i=1}^{n}X_i^2-5\overline{X_n^2}}=0.561,\\
&\widehat{\beta}_0=\Oyn-\widehat{\beta}_1\oxnn=331.16,\\
&\widehat{\sigma}^2=\frac{1}{5}\sum_{i=1}^{n}(Y_i-\widehat{\beta}_0-\widehat{\beta}_1X_i)^2=11.37\text{ a~nestranný }\\
& s^2=\frac{1}{5-2}\sum_{i=1}^{n}(Y_i-\widehat{\beta}_0-\widehat{\beta}_1X_i)^2=18.95.
\end{align*}
Spočítáme IS například pro~$\beta_1$. Dostaneme tedy $t_{0.975}(5-2)=3.18$, který dosadíme do~vzorečku na~výpočet IS pro~$\beta_1$, kde $\beta_1 \in (0.414,0.709)$.\\
 $\beta_1=0$, $
T_1=12.097$, $|T_1| \geq t_{0.975}(3)=3.18$, a~proto nezamítáme $H_0$.
\end{example}

%příklad matlab

\section{Intervaly predikce}
Předpokládejme, že máme nové pozorování $X$, pro~které je $Y$ neznámé a~my bychom chtěli predikovat hodnoty $Y$, případně najít intervaly spolehlivosti pro~$Y$. Vzhledem k~lineárnímu regresnímu modelu $Y=\beta_0+\beta_1 X + e$ je přirozené vzít za~predikci
$$\widehat{Y}=\widehat{\beta}_0+\widehat{\beta}_1X .$$
Najdeme rozdělení rozdílu $Y-\widehat{Y}$. Zřejmě se~jedná o~normální rozdělení ($\beta_0\sim \NN(...),\\ \beta_1 \sim \NN(...),e_1 \sim \NN(...), Y \sim \NN(...)$) stačí tedy určit střední hodnotu a~rozptyl.
$$\E(\widehat{Y}-Y)=\E(\widehat{\beta}_0)+\E(\widehat{\beta}_1X)-\beta_0-\beta_1 X-\E(e)=\beta_0+\beta_1 X-\beta_0-\beta_1 X -0=0. $$
Protože nový pár $(X,Y)$ je nezávislý na~předchozích datech, platí, že $Y$ je nezávislé na~$\widehat{Y}$ ($\beta_0,\beta_1$ jsou spočteny pouze pomocí $Y_1,...,Y_n$). Pak tedy
$$\D (\widehat{Y}-Y)=\D (\widehat{Y})+\D (Y)=\D (\widehat{Y})+\sigma^2,$$
protože $\D (Y)=\D (e)=\sigma^2$.
\[
\begin{split}
\D (\widehat{Y})&=\D (\widehat{\beta_0}+\widehat{\beta}_1X)=\E(\widehat{\beta}_0+\widehat{\beta}_1X-\beta_0-\beta_1 X)^2=\EE{\widehat{\beta}_0-\beta_0+X(\widehat{\beta}_1-\beta_1)}^2=\\
&=\underbrace{\E(\widehat{\beta}_0-\beta_0)^2}_{\D \widehat{\beta}_0}+\underbrace{X^2\E(\widehat{\beta}_1-\beta_1)}_{\D \widehat{\beta}_0}+2X\underbrace{\E(\widehat{\beta}_0-\beta_0)(\widehat{\beta}_1-\beta_1)}_{\D(\widehat{\beta}_0,\widehat{\beta}_1)}=\\
&=\Br{\frac{1}{n}+\frac{(\oxnn)^2}{x\sigma_X^2}}\sigma^2+X^2\frac{\sigma^2}{n\sigma_X^2}-2X\frac{\oxnn\sigma^2}{n\sigma_X^2}=\sigma^2\Br{\frac{1}{n}+\frac{(\oxnn-X)^2}{n\sigma_x^2}}
\end{split}
\]
Máme tedy
$$\widehat{Y}-Y \sim \n{0, \sigma^2\Br{1+\frac{1}{n}+\frac{(\oxnn-X)^2}{n\sigma_X^2}} },$$
a proto
$$ \frac{{(\widehat{Y}-Y)}\Big/{\sqrt{\sigma^2(1+\frac{1}{n}+\frac{(\overline{x_n}-X)^2}{n\sigma_x^2})}}}{\sqrt{\frac{1}{n-2}\frac{n\widehat{\sigma}^2}{\sigma^2}}}$$
a tedy $100(1-\alpha)$\% interval prediktu??? je
$$\widehat{Y}-t_{1-\frac{\alpha}{2}}(n-2)\sqrt{\frac{\widehat{\sigma}^2}{n-2}\Br{n+1+\frac{(\oxnn-X)^2}{\sigma_x^2}}} \leq Y \leq \widehat{Y}+t_{1-\frac{\alpha}{2}}(n-2)\sqrt{\frac{\widehat{\sigma}^2}{n-2}\Br{n+1+\frac{(\oxnn-X)^2}{\sigma_x^2}}}.$$
Tohle kreslí MATLAB (polytool)
\begin{example}[Rychlost zvuku]Mějme 
$\oxnn=30, ~\sigma_X^2=1760,~\widehat{\beta}_1=0.561,~\widehat{\beta}_0=331.16,~$\\$\sigma^2=11.37,~ \text{nestraný},~\widehat{s}^2=18.95$.
Nové $X=35^\circ C$ a~$\widehat{Y}=331.16+0.561\cdot 35=350.8$.
$$\sqrt{\frac{\widehat{\sigma}^2}{n-2}\Br{n+1+\frac{(\oxnn-X)^2}{\sigma_x^2}}}=\sqrt{\frac{11.37}{3}\Br{6+\frac{(30-35)^2}{1760}}}=4.77 $$
$$t_{0.975}(3)=3.1824 ~ \text{a tedy}~IP=(335.6,366.0)$$
\end{example}


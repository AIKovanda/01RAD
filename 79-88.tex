\section{Rezidua, diagnostika a transformace}
\begin{itemize}
 \item Je třeba ověřit adekvátnost modelu. Máme $ \text{R}^{2} , \text{t}, \text{F} $ statistiky, ty ale byly odvozeny za předpokladu linearity modelu a dalších podmínek na náhodné chyby. Pro ověření je důležitý nástroj analýza reziduí
 \item Je také třeba ověřit vliv jednotlivých pozorování na model - analýza odlehlých (outliers) a influenčních pozorování. (Velké reziduum pro $ i $-té pozorování naznačuje problém s modelem, ale může to být i naopak, vlivné pozorování nemusí mít velké reziduum)
 \item Pokud detekujeme nějaké problémy s modelem, mohou pomoci transformace proměnných nebo metoda na korekci nekonstantního rozptylu.
\end{itemize}
\subsection{Rezidua}
připomenutí:
$$
 \widehat{\textbf{y}} = \X \widehat{\beta} = \textbf{Hy} , \quad \text{kde} \quad \textbf{H} = \X ( \X ^{T} \X )^{-1} \X ^{T}
$$
$$
 \widehat{\textbf{e}} = \textbf{y} - \widehat{\textbf{y}} = ( \textbf{I}_n - \textbf{H} ) \textbf{y} =  ( \textbf{I}_n - \textbf{H} ) \textbf{e}
$$
Dále jsme ukázali:
$$
\E [ \widehat{\textbf{e}} ] = 0 \quad \text{Cov}(  \widehat{\textbf{e}} ) = \sigma^{2} ( \textbf{I}_n - \textbf{H} )
$$
Pokud navíc 
$
\textbf{e} \sim \text{N}_n ( 0 ,\sigma^{2} \textbf{I} ) $ potom $ \widehat{\textbf{e}} \sim \text{N}_n ( 0 ,\sigma^{2} ( \textbf{I}_n - \textbf{H})) 
$. \\ Pokud označíme $ h_{ii} = \textbf{H}_{ii} $, $ \widehat{\textbf{e}}_i \sim \text{N} ( 0 , \sigma^2 (1 - h_{ii} ) $, $ \text{Cov}( \widehat{\textbf{e}}_i , \widehat{\textbf{e}}_j  ) = -\sigma^2 h_{ij} $. \\
Obecně bývá vhodnější pracovat se standardizovanými rezidui, protože $ \D [  \widehat{\textbf{e}}_i ] = \sigma^2 (1 - h_{ii} ) $, pro $ r_i = \dfrac{\widehat{\textbf{e}}_i}{\sigma \sqrt{1-h_{ii}}} $ platí $ \D [ r_i ] = 1 $.
$ \sigma $ odhadneme pomocí $ s_n =  \sqrt{\frac{1}{n-m-1} \text{SSE}}$, dostaneme
$$
  \widehat{r}_i = \frac{\widehat{\textbf{e}}_i}{s_n \sqrt{1 - h_{ii}}} \quad \text{kde} \quad i = 1, \dots , n \quad \text{(Interně studentizované reziduum)}
$$
(Někdy také standardizovaná rezidua) \\
R : rstandard(.) \\
Pokud $ \sigma^2 $ odhadneme na základě modelu, ve kterém bylo vynecháno $ i $-té pozorování, označíme tento odhad $ \sigma^2_{(-i)} $, potom
$$
 \widehat{\text{t}}_i = \frac{\widehat{\textbf{e}}_i}{  \sigma^2_{(-i)} \sqrt{1 - h_{ii}}} \quad \text{kde} \quad i = 1, \dots , n \quad \text{(Externě studentizované reziduum)} 
$$
(Někdy také studentizované rezidua) \\
R : rstudent(.) \\
Například $ \sigma^2_{(-i)} = \dfrac{\text{SSE}_{(-i)}}{n-m-1} $ je nestranný odhad $ \sigma^2 $ v modelu $ (-i) $.
\begin{remark}
Platí:
\begin{itemize}
\item Pokud $ h_{ii} $ je malé, pro velké $ n $ by se mělo $ \widehat{\text{e}}_i, \widehat{\text{r}}_i, \widehat{\text{t}}_i $ chovat přibližně stejně a $  \widehat{\text{r}}_i ,\widehat{\text{t}}_i \approx \text{N} ( 0 , 1 ) $. 
\item Pro malé $ n $ $ ( n < 20 ) $ a / nebo  $ h_{ii} \approx 1 $ je preferováno použít $\widehat{\text{r}}_i$ nebo $\widehat{\text{t}}_i $ a aktuálně bývá častěji doporučována $\widehat{\text{t}}_i $ ( $ i $-té pozorování s velkými $ h_{ii} $ může zvyšovat odhad $ \sigma^2 $ a tím snižuje velikost svého rezidua).
\item $ h_{ii} $ hraje zásadní roli v diagnostice modelu, probereme teď jeho základní vlastnosti.
\end{itemize}
\end{remark}

\textbf{Leverage}$ h_{ii} $ - potenciál $ i $-tého pozorování (leverage point - píkový bod / vzdálený bod)

\begin{itemize}
\item $ \D [ \widehat{\text{e}}_i ] = \sigma^2 ( 1 - h_{ii} ) \geq 0 \Rightarrow h_{ii} \leq 1 $.
\item $ \textbf{H}^2 = \textbf{H} \Rightarrow h_{ii} = \sumjn h_{ij} h_{ji} = \sumjn (h_{ij})^2 $ tedy $ h_{ii} > 0 $ (Dá se ukázat silnější tvrzení: $ h_{ii} \geq \frac{1}{n} $).
\item $ \textbf{H}^2 = \textbf{H} \Rightarrow \textbf{A}_{i1} = \sumjn h_{ij} x_{j1} = \sumjn h_{ij} = x_{i1} = 1  $ tedy
$$ \sumjn h_{ij} = 1 \quad \forall j \in \widehat{n} $$.
\item Význam $ h_{ii} $ vyplyne z následujících úvahy:
$$ 
 \widehat{\textbf{y}} = \textbf{Hy} \Rightarrow \widehat{y}_i = \sumjn h_{ij}y_j = h_{ii} y_i + \sum\limits_{\substack{j=1 \\ i\neq j}}^n h_{ij} y_j
 $$
 pokud $ h_{ii} \approx 1 $, potom   $ \widehat{y}_i \approx y_i $ a model je nucen proložit přímku bodem ( $ \textbf{x}_i , y_i $) i když když tam neplatí body s "velkým $ h_{ii} $" - body s velkým potenciálem (high leverage points). Tyto body by měly být detekovány pro další zkoumání.
 \item Otázka je, jaká hodnota $ h_{ii} $ je "velká".
\end{itemize}
\textbf{Heuristické pravidlo:}
$ \sumin h_{ii} = \text{tr}(\textbf{H}) = m + 1 $, tzn. $ \dfrac{m+1}{n} $ je průměrná hodnota $ h_{ii}$ . $ i $-té pozorování má velký potenciál jestliže $ h_{ii} > \dfrac{3(m+1)}{n} $. (Stejně postupuje i jazyk R)

\subsection{Grafy reziduí}
\begin{enumerate}
\item Ověření normality - histogramy, Q-Q plots \\
tyto obrázky nezávisí na počtu nezávislých proměnných $ x $, vše stejné jako v jednoduché LR.
\item Pro ověření funkční formy pro $ \E [ Y_x ] $ a / nebo konstantního rozptylu se nejčastěji používají:
\begin{enumerate}
\item Grafy $ \widehat{e}_i , \widehat{r}_i $ nebo $ \widehat{t}_i $ oproti $ \textbf{x}_j^c $, $ j = 1, \dots , m $, kde $ \textbf{x}_j^c $ je $ j $-tý sloupec $ \X $
\item  Grafy $ \widehat{e}_i , \widehat{r}_i $ nebo $ \widehat{t}_i $ oproti $ \widehat{y}_i $
\item Partial residual plots
\end{enumerate}
\end{enumerate}

\begin{remark}
Zdůvodnění:
\begin{enumerate}
\item Normální rovnice $ \X ^{T} ( \textbf{y} - \X \widehat{\beta} ) = 0 $ implikují $ \X ^{T} ( \textbf{y} - \widehat{\textbf{y}}) = \X ^T \widehat{\textbf{e}} $.
$$
 \text{Připomenutí:} \quad \text{Y}_i = \beta_1 x_i + e_i , \quad \widehat{\beta}_1 = \dfrac{\sumin x_i y_i}{\sumin x_i^{2}} = \frac{\textbf{x}^{T}\textbf{y}}{\Vert \textbf{x} \Vert ^2 }
$$
Pokud tedy naladíme LR model bez interceptu pro $ \widehat{\textbf{e}} $ v závislosti na $ \textbf{x}_j^c $, směrnice přímky bude
$$
  \widehat{\beta}_j^* = \frac{(\textbf{x}_j^c)^T \widehat{\textbf{e}}}{\Vert \textbf{x}_j^c \Vert ^2} = 0
$$
Graf $ \widehat{e}_i , \widehat{r}_i , \widehat{t}_i $ oproti $ \textbf{x}_j^c $ by měl dávat náhodně rozptýlené body kolem osy $ x $. (bez trendů, $  \widehat{r}_i , \widehat{t}_i $ uvnitř ???? $\pm 2$ )
Pokud tomu tak není, může to naznačovat nelinearitu v $ \textbf{x}_j $ nebo nekonstantní rozptyl.
\item Ukázali jsme $ \sumin \widehat{y}_i \widehat{e}_i = 0 $ pro LM bez interceptu pro $ \widehat{e}_i $ oproti $ \widehat{y}_i $ tedy platí
$$
  \widehat{\beta} = \frac{\widehat{\textbf{e}}^T \widehat{\textbf{y}}}{\Vert \widehat{\textbf{y}} \Vert ^2} = 0
$$
Body by opět měly být náhodně rozptýlené kolem osy $ x $
\begin{itemize}
\item Trychtýřovitý tvar indikuje nekonstantní rozptyl.
\item Trendy indikují nelinearitu.
\end{itemize}
\end{enumerate}	
\end{remark}

\subsection{Partial residual plot}
\begin{itemize}
\item I když grafy $ \widehat{e}_i $ oproti $ \textbf{x}_j^c $ a $ \widehat{\textbf{y}} $ mohou indukovat nedostatky modelu, nemusí být zřejmé, jaké ty nedostatky jsou.
\item V SLR graf $ \widehat{e}_i $ oproti $ x_i $ lze použít pro detekci nelinearity
\item Ale v MLR tyto grafy, stejně jako scatterploty, mohou být zavádějící, protože $ \widehat{\textbf{e}} $ závisí na všech prediktorech, nemusí být tedy izolován efekt dané proměnné při odstranění efektů ostatních.
\item Pro zkoumané efekty $ j $-té proměnné lze použít partial rezidual plots - lze je chápat jako jeho ekvivalent scatterplotu v SLR

\begin{define}
 $$
   \widehat{e}_j^* = \widehat{\textbf{e}} + \widehat{\beta}_j \textbf{x}_j^c,
 $$
 kde $ \widehat{\textbf{e}} $ je vektor reziduí modelu, $ \widehat{\beta}_j $ je LSE parametru $ \beta_j $ , $ \textbf{x}_j^c $ je $ j $-tý sloupec $ \X $
\end{define}
\end{itemize}
Partial residual plot (PRP): graf $ \widehat{\textbf{e}} $ oproti $ \textbf{x}_j^c $ , $ j = 1, \dots , m $ pokud je model správný, měly by být body náhodně rozmístěné kolem přímky se směrnicí $ \widehat{\beta}_j $.

\textbf{Zdůvodnění}: Vztah mezi $\heb_j^{*}$ a $\x_j^c$ má formu SLR bez interceptu, pokud je model správný, $\hei, i = 1, \dots, n$, splňují podmínku $\E \hei = 0$ a $\D \hei = \sigma^2(1 - h_{neco})$. Má tedy smysl uvažovat RM pro $\heb_j^{*}$ oproti $\x_j^c$ ($\he_j^{*} = \gamma_j \x_j^c + \eb$).

\newcommand{\hg}{\widehat{\gamma}}

Pro odhad koeficientů platí:
$$
\hg_j = \frac{(\heb_j^{*} \x_j^c)}{\norm{\x_j^c}^2} = \frac{(\heb + \wbeta_j \x_j^c)^T \x_j^c}{\norm{\x_j^c}^2} = \frac{\heb^T \x_j^c + \wbeta_j \norm{\x_j^c}^2}{\norm{\x_j^c}^2} = \wbeta_j,
$$
protože $\heb^T \x_j^c = 0$.

(2 příklady - pdf 79-93 uprostřed str 6)

\newcommand{\wb}{\mathbf{w}}
\newcommand{\identita}{\mathbb{I}}
\newcommand{\Xmj}{\X_{(-j)}}
\newcommand{\Xmi}{\X_{(-i)}}

\begin{remark}
	PRPs jsou někdy kritizovány za nadhodnocování efektu $\x_j^c$. \linebreak
	Alternativa: \textbf{partial regression plot (added variable plot)}. \linebreak
	Motivace: Ptáme se, zda přidat novou proměnnou do modelu a chtěli bychom dohadnout její efekt.
	
	Budeme tedy uvažovat rozšířený model
	$$
	\Y = \X \betab + \gamma \wb + \eb,
	$$
	kde $\wb$ je nový vektor regresorů. Model lze rozepsat jako
	$$
	\Y =  \begin{bmatrix}
	\X \wb
	\end{bmatrix} \begin{pmatrix}
	\betab \\ \gamma
	\end{pmatrix} + \eb = \X_{w} + \betab_{w} + \eb.
	$$
	Použitím normálních rovnic pro $\X_{w}$ lze odvodit formuli pro $\hg$
	\begin{equation*}
		\hg = \frac{\heb^T (\identita - \Hm) \wb}{\norm{(\identita - \Hm)\wb}^2}.
		\tag{\#}
		\label{Eq: hat gamma}
	\end{equation*}
	$\hg$ je směrnice RM pro $\heb$ v závislosti na $\wb_{res} = (\identita - \Hm)\wb$ (rezidua modelu pro $\wb$ v závislosti na $\X$).
	
	Teď naopak uvažujme, že $\wb$ je sloupec původní $\X$, řekněme $\x_j^c$ a ozn. $\Xmj$ matici $\X$ bez sloupce $j$. V předchozím modelu pomožme $\X = \Xmj$ a $\wb = \x_j^c$. Potom  LSE $\wbeta_j$ parametru $\beta_j$ je
	$$
	\wbeta_j = \frac{\heb_{(-j)}^T \x_{j, res}^c}{\norm{\x_{j,res}^c}^2},
	$$
	kde $\heb_{(-j)}$ jsou rezidua modelu bez $\x_j^c$, $\x_{j,res}^c = (\identita - \Hm) \x_j^c$, tedy jsou to rezidua modelu pro $\x_j^c$  v závislosti na ostatních proměnných, tedy $\Xmj$ (v $\x_{j,res}^c$ je tedy odstraněn efekt ostatních regresorů).
	
	$\wbeta_j$ je směrnice RM pro $\heb_{(-j)}$ v závislosti na $\x_{j, res} \implies$ \linebreak
	\textbf{added variable plot}: graf $\heb_{(-j)}$ proti $\x_{j, res}, j = 1, \dots, m$.
	
	Pokud je model správný, body by měly být náhodně rozptýlené kolem přímky se směrnicí $\wbeta_j$ procházející počátkem. Pokud závislost na $\x_j^c$ není lineární, projeví se to odklonem bodů od přímky.
\end{remark}

\begin{remark}
Ze vztahu \eqref{Eq: hat gamma} je vidět, že MLR může být chápána jako posloupnost SLR, kde postupně vytváříme modely pro novou proměnnou s použitím reziduí modelu pro předcházející proměnné.
\end{remark}

\subsubsection{PRESS rezidua (PRESS residuals, deleted residuals)}
\begin{itemize}
	\item pokud budeme chtít model použít nejen k vysvětlení vztahu mezi proměnnými, ale také pro predikci, hodila by se míra vyjadřující jak dobře model predikuje (doposud jsme zkoumali jen jak dobře popisuje)
	\item šlo by použít IS nebo IP, to bychom ale předem museli znát body, ve kterých chceme predikovat
	\item nejjednodušší přístup, jak měřit prediktivní přesnost modelu by byl analýza reziduí pro predikce hodnot v nových bodech $\x$, obecně ale nemáme data $y$ v těchto bodech
	\item jedna možnost je použít data, která máme k dispozici
\end{itemize}

\textbf{Postup}: Vynecháme jedno pozorování, naladíme model bez tohoto pozorování a porovnáme predikovanou a pozorovanou hodnotu pro vynechané pozorování.

Předpokládáme, že vynecháme $i$-té pozorování. Ozn. $\wbetab_{(-i)}$ odhad $\betab$ v modelu s vynechaným $i$-tým pozorováním (M$_{-i}$) a $\hy_{(-i)}$ predikovanou hodnotu modelem M$_{(-i)}$ v bodě $\x_i^T$, tzn. $\hy_{(-i)} = \x_i^T \wbetab_{(-i)}$.

Potom
$$
\heb_{(-i)} = y_i - \hy_{(-i)}, \quad i = 1, \dots, n
$$
nazýváme $i$-té \textbf{PRESS reziduum}.

$\mathrm{PRESS} = \sumin \he_{-i}^2$ je užitečná míra přesnosti predikce.

\begin{remark}
	Otázka je, jak počítat $\he_{(-i)}, i = 1, \dots, n$.
	\begin{itemize}
		\item pro velké $n$ se zdá, že to bude náročný problém, protože pro každé $i \in \hat{n}$ musíme naladit nový model
		\item naštěstí to není nutné, ukážeme totiž, že
		$$
		\he_{(-i)} = \frac{\he_i}{1 - \hii},
		$$
		tzn. všechna $\he_{(-i)}$ lze snadno spočítat pomocí reziduí a hodnot $\hii$ z původního (plného) modelu.
	\end{itemize}
\end{remark}

Označme
\begin{align*}
	\x_i^T & \text{ - $i$-tý řádek matice } \X \\
	\Xmi & \text{ - matici $\X$ bez $i$-tého řádku}
\end{align*}

\newcommand{\XtX}{\left( \X^T \X \right)^{-1}}

\begin{theorem}
Jestliže $\hii \neq 1$, potom
$$
\left[ \Xmi^T \Xmi \right] = \X^T \inv{\X} + \frac{\XtX \x_i \x_i^T \XtX}{1 - \hii},
$$
kde $\hii$ je $i$-tý diagonální prvek matice $\Hm$.
\end{theorem}

\begin{proof}
Nejdříve ukážeme
\begin{equation*}
	\X^T \X = \Xmi^T \Xmi + \x_i \x_i^T \tag{+} \label{Eq: plus}
	%(m+1) \times (m+1) & (m+1) \times (m+1) & (m+1) \times (m+1)
\end{equation*}

\newcommand{\sumkn}{\sum_{k=1}^n}
\newcommand{\sumknn}{\sum_{k=1}^{n-1}}

Kvůli značení předpokládáme $i = n$ (toho se dá vždy dosáhnout permutací řádků $\X$). Potom
$$
\left( \X^T \X \right)_{ij} = \sumkn x_{ki} x_{kj} = \sumknn x_{ki} x_{kj} + x_{ni} x_{nj}.
$$
$i,j$-tý prvek $\X_{(-k)}^T \X_{(-k)}$ je $\sumknn x_{ki} x_{kj}$ \\
$i,j$-tý prvek $\x_n \x_n^T$ je $x_{ni} x_{nj}$, tzn. \eqref{Eq: plus} platí.


\newcommand{\z}{\mathbf{z}}

\begin{theorem}[Sherman-Morrison-Woodburry (z LA)]
Nechť $\Am$ je $n \times n$ invertibilní matice a nechť $\z$ je $n \times 1$ sloupcový vektor. Jestliže $\z^T \inv{\Am} \z \neq 1$, potom matice $\Bm = \Am - \z^T \z$ je invertibilní a platí
$$
\inv{\Bm} = \inv{\Am} + \frac{\inv{\Am} \z^T \z \inv{\Am}}{1 - \z^T \inv{\Am} \z}.
$$
\end{theorem}

Položme $\Am = \X^T \X$, $\z = \x_i$, $\Bm = \X_{(-i)}^T \X_{(-i)}$. Pak $\Bm = \Am - \z \z^T$, $\Am$ je invertibilní a
$$
\z^T \inv{\Am} \z = \x_i^T \XtX \x_i = \left( \X \XtX \X^T \right)_{ii} = \hii \neq 1.
$$
Užitím věty dostaneme
$$
\left(  \X_{(-i)}^T \X_{(-i)} \right)^{-1} = \XtX + \frac{\XtX \x_i \x_i^T \XtX}{1 - \hii}.
$$

\end{proof}

\begin{theorem}
	Nechť $\he_{(-i)}$ je $i$-té PRESS reziduum. Potom
	$$
	\he_{(-i)} = \frac{\he_i}{1 - \hii}, \quad i = 1, \dots, n.
	$$
\end{theorem}

\begin{proof}
	
	Nechť $\wbetab_{(-i)}$ je odhad $\betab$ v modelu M$_{-i}$, tzn.
	$$
	\wbetab_{(-i)} = \left( \X_{(-i)}^T \X_{(-i)} \right)^{-1} \X_{(-i)}^T \y_{(-i)},
	$$
	tedy $\y_{(-i)}$ je $\y$ bez $i$-té složky $y_i$. Tzn.
	\begin{align*}
	\hy_{(-i)} & = \x_i^T \wbetab_{(-i)} = \x_i^T \inv{\left( \X_{(-i)}^T \X_{(-i)} \right)} \X_{(-i)}^T \y_{(-i)} = \\
	 & = \left[ \left(  \X_{(-i)}^T \X_{(-i)} \right)^{-1} = \XtX + \frac{\XtX \x_i \x_i^T \XtX}{1 - \hii}. \right] = \\
	 & = \x_i^T \XtX \X_{(-i)}^T \y_{(-i)} + \frac{1}{1 - \hii} \x_i^T \XtX \x_i \x_i^T \XtX \X_{(-i)}^T \y_{(-i)} = \\
	 & = S_1 + \frac{1}{1-\hii} S_2.
	\end{align*}
	
	Protože $\X_{(-i)}^T \y_{(-i)} = \X^T \y - y_i \x_i$, dostaneme
	\begin{align*}
	S_1 & = \x_i \XtX \left( \X^T \y - y_i \x_i \right) = \x_i^T \underbrace{\XtX \X^T \y}_{\wbetab} - y_i \underbrace{\x_i^T \XtX \x_i}_{\hii} = \\
	& = \x_i^T \wbetab - \hii y_i = \hy_i - \hii y_i.
	\end{align*}
	Podobně	
	$$ S_2=\underbrace{\textbf{x}_i^T(\textbf{x}^T\textbf{x})^{-1}\textbf{x}_i}_{h_{ii}}\underbrace{\textbf{x}_i^T(\textbf{x}^T\textbf{x})^{-1}\textbf{x}\textbf{y}}_{\hyi}=y_i\underbrace{\textbf{x}_i^T(\textbf{x}^T\textbf{x})^{-1}\textbf{x}_i}_{h_{ii}}\underbrace{\textbf{x}_i^T(\textbf{x}^T\textbf{x})^{-1}\textbf{x}_i}_{h_{ii}}=h_{ii}\hyi-y_ih_{ii}^2 $$
	takže 
	$$ \hy_{(-i)}=\hyi-h_{ii}y_i+\frac{1}{1-h_{ii}}(h_{ii}\hyi-y_ih_{ii}^2).$$
	Celkem tedy 
	\begin{align*}
	\he_{-i} & =y_i-\hy_{(-i)}=y_i(1+h_{ii})-\hyi-\frac{1}{1-h_{ii}}(h_{ii}\hyi-y_ih_{ii}^2)= \\
	&=\frac{1}{1-h_{ii}}\big(y_i(1-h_{ii}^2)-\hyi(1-h_{ii})-h_{ii}\hyi+y_ih_{ii}^2\big)=\frac{1}{1-h_{ii}}(y_i-\hyi)=\frac{\hei}{1-h_{ii}}
	\end{align*}
\end{proof}
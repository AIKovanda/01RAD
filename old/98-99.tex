Pro odstranění vztahu $ \E [\text{Y}] $ a $ \D[\text{Y}] $ se často používají mocninné transformace $ y^* = y^{\lambda} $ (pro $ y > 0 $).

\begin{table}[h]
\centering
$\begin{array}{ *{13}{c} }
\text{Transformace:} & \leftarrow & \dots &  y^3 &  y^2 &  y & \sqrt{y}  & \text{log}y & \dfrac{1}{\sqrt{y}}  & \dfrac{1}{y} & \dfrac{1}{y^2} & \dots  & \rightarrow \\
\text{Box-Cox} \lambda : &\leftarrow  & \dots & 3 & 2 & 1 & \dfrac{1}{2}  & 0 &  -\dfrac{1}{\sqrt{2}} & -1 & -2 & \dots & \rightarrow
\end{array}$
\end{table}

\begin{itemize}
\item Pokud $ \D[\text{Y}] $ klesá s rostoucí $ \E[\text{Y}] $, budeme klesat s mocninou $\lambda$.
\item Pokud $ \D[\text{Y}] $ roste s rostoucí $ \E[\text{Y}] $, budeme $\lambda$ snižovat.
\end{itemize}

OBECNĚ: Předpokládejme vztah $ \D[\Y] = \phi \text{V}(\mu) $ a uvažujeme transformaci $ y^* = \text{h}(y) $. Taylorův rozvoj 1. řádu funkce $ \text{h}(y) $ v bodě $ \mu $
$$  
 y^* = \text{h}(y) \approx \text{h}(\mu) + \text{h}'(\mu)(y-\mu)
$$
z čehož plyne, že $ \D[\text{Y}^*] \simeq ( \text{h}'(\mu) )^2 \cdot \D[\text{Y}] $
Transformace $  y^* = \text{h}(y) $ tedy bude přibližně stabilizovat rozptyl, pokud $ \text{h}'(y) $ je úměrné $ \left( \D[\text{Y}]^{-1/2} = \text{V}^{-1/2}(\mu) \right)$.

\begin{itemize}
\item Pokud $ \text{V}(\mu) = \mu^2 \quad \Rightarrow \quad $ stabilizující transformace je $ \text{log}(y) = \text{h}(y) $ protože $ \text{h}'(y) = \dfrac{1}{\mu} $.
\item Pokud $ \text{V}(\mu) = \mu \quad \Rightarrow \quad $ stabilizující transformace je $ \text{h}(y) = \sqrt{y} $ protože $ \text{h}'(y) = \dfrac{1}{2 \sqrt{\mu}} $.
$$  
 \left( \text{h}(\mu) =  \int \frac{\d \mu}{\sqrt{\text{V}(\mu)}} \right)
$$
\item Asi nejvíce užívanou transformací je $ y^* = \text{log}(y) $. Jedním z důvodů je i dobrá interpretovatelnost parametru $ \beta $.
\end{itemize}

Interpretace parametrů LM

\begin{enumerate}
\item Klasický LM: 
$$
 \E [ Y ] = \beta_0 + \beta_1 x_1 + \dots + \beta_m x_m.
$$
Jednorozměrná změna proměnné $ x_j \, \Rightarrow $ změnu $ \E [ Y ] $ o $ \beta_j $ jednotek (při ostatních proměnných stálých).

$$
\left( \begin{matrix}
\X = (1,x_1,\dots,x_m) & \X_{\text{new}} = (1,x_1,\dots,x_j+1,\dots,x_m) \\
\downarrow &  \downarrow \\
\E [ Y ] & \E [ Y_{\text{new}} ] \\
 & \\
 \E [ Y_{\text{new}} ] - \E [ Y ]  = \beta_j &
\end{matrix} \right)
$$
\item LM pro $\log Y$:
$$
 \log Y = \beta_0 + \beta_1 x_1 + \dots + \beta_m x_m + e \quad, \text{kde} \quad e \sim \NN(0,\sigma^2).
$$
Pokud je to správný model, znamená to, že $ \log Y \sim \NN(\mu,\sigma^2) \, \Rightarrow \, Y \sim \mathcal{LN}(\mu,\sigma^2) $ a tedy $ \E[Y] = e^{\mu + \frac{\sigma^2}{2}} $. \\
- Predikce pro $ \E [\log Y] $ je $ \widehat{\mu} = \wbeta_0 + \wbeta_1 x_1 + \dots + \wbeta_m x_m $. \\
- Predikce pro $ \E [Y] $ bude $ e^{\wbeta_0 + \wbeta_1 x_1 + \dots + \wbeta_m x_m + \frac{\wsigma^2}{2}} $.

Uvazujme opět jednotkovou změnu p. $ x_j $ ($ x_j \rightarrow x_j + 1 $):
$$
\dfrac{ \E [ Y_{\text{new}} ]}{\E [ Y ]} = \dfrac{e^{\wbeta_0 + \wbeta_1 x_1 + \dots + \wbeta_j x_j + \wbeta_j + \dots + \wbeta_m x_m + \frac{\sigma^2}{2}}}{e^{\wbeta_0 + \wbeta_1 x_1 + \dots + \wbeta_m x_m + \frac{\sigma^2}{2}}} = e^{\wbeta_j}
$$
Pak jednotková změna proměnné $ x_j \, \Rightarrow  $ multiplikativní změna $ \E [ Y ] \, e^{\wbeta_j}\text{-krát} $. Jinak zapsáno: $ 100(e^{\wbeta_j}-1) $ je procentní změna $ \E [ Y ]  $ spojená s jednotkovou změnou $ x_j $
\end{enumerate}
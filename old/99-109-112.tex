\begin{itemize}
	\item z dosazení vyplývá, že odhady parametru $\betab$ a $\sigma^2$ lze získat použitím transformovaného modelu (2)
	\item Protože ale transformovaný model neobsahuje intercept (první sloupec M je $(\sqrt{w_1}, \dots, \sqrt{w_n})^T)$), nefunguje klasický rozklad součtu čtverců a F statistika nelze definovat obvyklým způsobem, stejně jako $R^2$ (viz. regrese skrz počátek)
	\item nicméně princip \uv{extra sum of squares} funguje, ať má model intercept nebo ne:
	např. celkový F-test lze provést pomocí statistiky
	$$
	F_w = \frac{\SSE_R - \SSE_F}{\frac{m}{s_w^2}},
	$$
	kde $\SSE_F$ je reziduální součet čtverců $s_w^2$ plného modelu a $\SSE_R$ je reziduální součet čtverců redukovaného transformovaného modelu $\Z = \Mm_0 \betab_0 + \eb$, $\Mm_0 = (\sqrt{w_1}, \dots, \sqrt{w_n})^T)$.
	
	Pokud mají chyby normální rozdělení, platí za $H_0: \beta_1 = \cdots = \beta_m = 0$, že $F_w \sim F(m, n-m-1)$ a $H_0$ zamítáme, pokud $F_w > F_{1-\alpha}(m,n-m)$.
	\item str. 110 (*)
	\item pro analýzu reziduí je třeba uvažovat vhodné grafy reziduí:
	\begin{itemize}
		\item máme dva vektory reziduí:
		\begin{align*}
			& \hei \text{ v původním modelu (1)} \\
			& \heps_i \text{ v transformovaném modelu (2)}
		\end{align*}
		a tedy dvě možnosti
		\item pro kontrolu konstantního rozptylu lze uvažovat i standardizovaná nebo studentizovaná rezidua (pomocí bodu 4) a 5) věty lze ukázat, že jsou v obou modelech stejná)
		\item je třeba být opatrný oproti jakým hodnotám budeme rezidua zobrazovat
		\item grafy $\heps_i$ proti sloupcům $\Mm$ a predikovaným hodnotám $\hz$ jsou OK, neboť např.
		$$
		\sumin \hz_i \heps_i = 0
		$$
		(jsou OG, měl by být vidět roztýlený oblak kolem osy x).
		\item dosazením $\heps_i = \sqrt{w_i}  \cdot \hei$ a $\hz_i = \sqrt{w_i} \cdot \hy_i$ dostaneme $\sumin w_i \hei \hy_i = 0$, tzn. graf $\hei$ proti $\hy_i$ bude zavádějící
		\item graf $\sqrt{w_i}  \cdot \hei$ proti $\sqrt{w_i}  \cdot \hy_i$ je ale v pořádku
		\item podobné závěry platí i pro grafy $\hei$ proti $\x_j^c, i = 1, \dots, m$.
 	\end{itemize}
 	\item (*): přirozené je definovat $R^2 = \rho^2(\hzb, \z)$, kde $\rho(\hzb, \z)$ je výběrový korelační koeficient, pro $\mathbb{W} = \I$ dostaneme standardní $R^2$
\end{itemize}

\begin{remark}
\begin{itemize}
	\item pokud jsou váhy neznámé, bylo by třeba je odhadnou společně s $\betab$ a $\sigma^2$ z dat
	\item to ale není obecně možné, protože máme více parametrů než dat
	\item někdy to možné je, pokud máme další informace o rozdělení chyb (tvar kovarianční matice atd.)
\end{itemize}
\end{remark}

\begin{remark}
	Celý postup WLS lze použít i na případ $\eb \sim \NN_m(0,\sigma^2 \Wm)$, kde $\Wm$ je známá, ale není diagonální. Protože $\Wm$ je symetrická, ex. regulární $\Km$ tak, že $\Wm = \Km \Km^T$. Stejná transformace jako u WLS opět vede na transformovaný model, kde $\epsb \sim \NN_m(0,\sigma^2 \Identita{m})$.
\end{remark}

\section{Korelované chyby}
\begin{itemize}
\item Zejména v časových nebo ekonomických datech se často objevuje korelace jednotlivých hodnot.
\item potom není splněn předpoklad nezávislosti chyb
\item tento stav je třeba detekovat (někdy pomohou grafy reziduí)
\item modely pro korelovaná data: \textbf{Analýza časových řad}
\end{itemize}

Pokud je přítomna autokorelace a chyby mají konstantní rozptyl, platí:
\begin{enumerate}
\item OLS odhad $\wbeta$ je nestranný, ale neplatí Gauss-Markovova věta, tzn. $\wbeta$ nemá nejmenší rozptyl.
\item MSE = $\frac{1}{n-m-1}\SSE$ (odhad $\sigma^2$) může být podstatně menší než skutečná hodnota $\sigma^2$, což může dávat falešný pocit přesnosti.
\item V důsledku bodu 2) mohou být zvětšeny hodnoty T statistik, takže testy o parametrech a IS nefungují.
\item Protože jsou chyby nezávislé, F-testy a t-testy nejsou přesně platné ani když jsou chyby normální.
\end{enumerate}

\subsection{Durbin-Watson statistika}
Omezíme se na pozorování získaná v čase $t = 1,2,\dots, n$ a případ, že chyby $e_t$ splňují podmínky autoregresního procesu 1. řádu (AR1), tj.
$$
e_t = \rho e_{t-1}+ u_t, \quad |\rho| < 1,
$$
kde $\rho$ je autokorelační koeficient, $u_t \sim \NN (0, \sigma_n^2)$ jsou nezávislé v $t = 1,\dots, n$ a $u_t$ je nezávislá na $e_t, t \geq 1$. Častěji pro data časových řad platí $\rho > 0$ (pozitivní autokorelace).

Pro test $\hypothesis{\rho = 0}{\rho > 0}$ se užívá \textbf{Durbin-Watsonova statistika}
$$
d = \frac{\sum_{t=2}^n(\he_t - \he_{t-1})^2}{\sum_{t=1}^n \he_t^2},
$$
kde $\he_t$ jsou rezidua modelu LR. Pokud zamítneme $H_0$, odhadne se $\rho$ pomocí
$$
\widehat{\rho} = \frac{\sum_{t=2}^n \he_t \he_{t-1}}{\sum_{t=2}^n \he_t^2}.
$$

\newcommand{\sumtn}{\sum_{t=2}^{n}}
\begin{remark}
Dá se ukázat, že $d \approx 2(1-\hrho)$:
$$
\sum_{t=2}^n(\he_t - \he_{t-1})^2 = \sumtn \he_t^2 + \sumtn \he_{t-1}^2 - 2 \sumtn \he_t \he_{t-1} \approx 2 \left( \sumtn \he_t^2 - \sumtn \he_t \he_{t-1} \right),
$$
Z Cauchy-Schwartzovy nerovnosti $\implies \frac{\sumtn \he_t \he_{t-1}}{\sum_{t=1}^n \he_t^2}$ leží přibližně v $(-1,1)$, tzn. $d$ leží přibližně v $(0,4)$. Dále
$$
\hrho \approx 1 \implies d \approx 0 \quad \text{a} \quad \hrho \approx 0 \implies d \approx 2,
$$
tzn. pro malé hodnoty $d$ budeme zamítat $H_0$, pro velké hodnoty nebudeme zamítat. Kritické hodnoty určené Durbinem a Watsonem jsou tabelované.
\end{remark}

\noindent \textbf{Test:}
\begin{enumerate}
	\item spočítat hodnotu $d$
	\item nalézt kritické hodnoty $(d_L,d_U)$ pro dané $n$ a $m+1$
	\item \begin{enumerate}
		\item zamítnout $H_0$, pokud $d < d_L$
		\item nezamítnout $H_0$, pokud $d > d_U$
		\item pro $d_L < d < d_U$ test nerozhodne
	\end{enumerate}
\end{enumerate}

\begin{remark}
	Pro test $\hypothesis{\rho = 0}{\rho < 0}$ lze použít popsaný test pro $d' = 4 - d$. Metody pro korekci autokorelace: \textbf{Cochrane-Orcutt}.
\end{remark}
\newcommand{\hx}{\bar{x}}
\newcommand{\hej}{\widehat{e}_j}
\newcommand{\Nn}{\mathcal{N}(0,\sigma^2)}
\newcommand{\hZi}{\widehat{Z}_i}
\newcommand{\hzi}{\widehat{z}_i}
\newcommand{\model}{\wbeta_0 + \wbeta_1 x_i}
\newcommand{\hYj}{\widehat{Y}_j}
\newcommand{\hr}{\widehat{r}}

\begin{theorem}
	Nechť $\hei$ jsou rezidua modelu $(*)$ odhadnutého metodou nejmenších čtverců. Potom platí:
	\begin{enumerate}
		\item $\E \hei = 0, \quad i = 1, \dots, n$
		\item $\D \hei = \sigma_{}^2 = \sigma^2 \left[ 1 - \left( \frac{1}{n} + \frac{(x_i - \hx)^2}{\Sxx} \right) \right] \approx \sigma^2$ pro velká $n$
		\item $\Cov (\hei, \hej) = -\sigma^2 \left[ 1 - \left( \frac{1}{n} + \frac{(\hx - x_i)(\hx - x_j)}{\Sxx} \right) \right]$
		\item $\Cov(\hei, \hYi) = 0 = 0, \quad i = 1, \dots, n$
		\item Pokud jsou $e_1, \dots, e_n$ iid $\Nn$, potom platí:
		$$
			\hZi = \frac{\hei}{\sigma_{\hei}} \sim \NN (0,1).
		$$
	\end{enumerate}
\end{theorem}

\begin{proof}
\begin{enumerate}
	\item $\hei = Y_i - \hYi$, takže $\E(\hei) = \E Y_i - \E \hYi$, ale $\E \hYi = \E(\model) = \model = \E Y_i$
	\item 
	$$
		\D \hei = \D (Y_i - \hYi) = \D Y_i + \underbrace{\D \hYi}_{\sigma^2 \left[ \frac{1}{n} + \frac{(x_i - \hx)^2}{\Sxx} \right]} - 2 \underbrace{\Cov(Y_i, \hYi)}_{\sigma^2 \left[ \frac{1}{n} + \frac{(x_i - \hx)^2}{\Sxx} \right]}  = \sigma^2 \left[ 1 - \left( \frac{1}{n} + \frac{(x_i - \hx)^2}{\Sxx} \right) \right]
	$$
	\item 
	\begin{align*}
		\Cov(\hei, \hej) & = \Cov(Y_i - \hYi, Y_j - \hYj) = \underbrace{\Cov(Y_i, Y_j)}_{ = 0} - \Cov(Y_i, \hYj) - \Cov(Y_i, \hYj) + \Cov(\hYi, \hYj) \\
		\Cov(\hYi, \hYj) & = \Cov(\model, \wbeta_0 + x_j \wbeta_1) = \underbrace{\D(\wbeta_0)}_{\sigma^2 \left[ \frac{1}{n} + \frac{\hx^2}{\Sxx} \right]} + (x_i + x_j) \underbrace{\Cov (\wbeta_0,\wbeta_1)}_{-\frac{\sigma^2 \hx}{\Sxx}} + x_i x_j \underbrace{\D(\wbeta_1)}_{\frac{\sigma^2}{\Sxx}} = \\
		& = \sigma^2 \left[ \frac{1}{n} + \frac{\hx^2}{\Sxx} - \frac{(x_i + x_j) \hx}{\Sxx} + \frac{x_i x_j}{\Sxx}\right] = \sigma^2 \left[ \frac{1}{n} + \frac{(x_i - \hx)(x_j - \hx)}{\Sxx} \right]
	\end{align*}
	
	Podobně bychom dostali
	$$
		\Cov(Y_i, \hYj) + \Cov(\hYi, Y_j) = 2 \sigma^2 \left[ \frac{1}{n} + \frac{(x_i - \hx)(x_j - \hx)}{\Sxx} \right]
	$$
	takže $\Cov(\hei, \hej) = - \sigma^2 \left[ \frac{1}{n} + \frac{(x_i - \hx)(x_j - \hx)}{\Sxx} \right]$.
	\item 
	$$
		\Cov(\hei, \hYi) = \Cov(Y-i - \hYi, \hYi) = \underbrace{\Cov(Y_i, \hYi)}_{ = \sigma^2 \left[ \frac{1}{n} + \frac{(x_i - \hx)^2}{\Sxx} \right]} - \underbrace{\D(\hYi)}_{ = \Cov(\hYi, \hYi) = \sigma^2 \left[ \frac{1}{n} + \frac{(x_i - \hx)^2}{\Sxx} \right]} = 0
	$$
	\item $e_i \sim \Nn \implies \hei \sim \NN(\cdot, \cdot)$, protože $\hei$ je LK $Y_1, \dots, Y_n$
	\begin{itemize}
		\item 1) $\implies \E \hei = 0$
		\item 2) $\implies \D \hei = \sigma_{\hei}^2$
	\end{itemize}
	$\implies \frac{\hei}{\sigma_{\hei}} \sim \NN(0,1)$
\end{enumerate}
\end{proof}

\begin{remark}
	Z bodu 3) věty plyne, že $\Cov(\hei, \hej) \approx 0$ pro velké $n$. Pokud jsou testy $e_i$ iid $\Nn$, měla by se standardizovaná rezidua $\hZi = \frac{\hei}{\sigma_{\hei}}$ chovat pro velké $n$ jako náhodný výběr z $\NN(0,1)$ rozdělení. V praxi ale budeme potřebovat odhad $\sigma^2$ pro výpočet $\hZi$.
	
	Nejznámější procedura: odhadnout $\sigma^2$ pomocí $s_n^2$, potom
	$$
		\hzi = \frac{\hei}{s_n \sqrt{1 - \left( \frac{1}{n} + \frac{(x_i - \hx)^2}{\Sxx} \right)}} \quad \text{standardizovaná rezidua}
	$$
	by se opět pro velká $n$ měla chovat jako NV z $\NN(0,1)$.
\end{remark}

\begin{remark}
	$\hei$ se užívají pro grafickou analýzu.
	
	Jiná třída reziduí -- PRESS rezidua (? metody zkoumání reziduí):
	
	ozn. $\wbeta_{0(-i)}, \wbeta_{1(-i)}$ odhady parametrů $\beta_0, \beta_1$, pokud je vynecháno i-té pozorování. Pak i-té PRESS reziduum je definováno jako
	$$
		\widehat{e}_{(-i)} = \hYi - \widehat{Y}_{(-i)}, \quad \text{kde \;} \widehat{Y}_{(-i)} = \wbeta_{0(-i)} + x_i \wbeta_{1(-i)}.
	$$
	Podrobněji se jim budeme věnovat později.
\end{remark}

\section{Grafy reziduí}
\begin{itemize}
	\item Histogram reziduí (náhled normality reziduí).
	\item Kvantilový graf (QQ plot) standardizovanách reziduí -- seřadíme dle velikosti: $\hr_{(1)} \leq \hr_{(2)} \leq \dots \leq \hr_{n}$ a vyneseme oproti $\Phi^{-1}\left( (i - \frac{1}{2}) \frac{1}{n} \right)$, $i = 1, \dots, n$. Body by měly ležet přibližně na přímce ($\E (e_{i} ) \approx \Phi^{-1}\left( (i - \frac{1}{2}) \frac{1}{n} \right)$ pro normální chyby).
	
	Použití: ověření normality, detekce odlehlých pozorování (obr. 3.6 str. 1077 GLM).
	\item Standardizovaná rezidua $\times$ jednotlivým vysvětlujícím proměnným $x$ -- $\hr_i$ nezávisí na $\sigma$, graf $\hr_i \times x_i$ lze použít pro detekci nelinearity nebo nekonstantního rozptylu.
	\item Standardizovaná rezidua $\hr_i$ $\times$ predikovaným hodnotám $\hyi$ -- $\Cov(\hei, \hYi) = 0$, tedy $\hei(\hr_i)$ a $\hYi$ by měly být nekorelované, pokud platí model $(*)$. Tzn. graf $\hr_i \times \hyi$ by měl být náhodně rozptýlený kolem osy $x$, navíc $\hr_i$ by měla ležet v $(-3,3)$ ($\hr_i \approx \NN(0,1)$).
	
	Obrázky....
	
	\item Standardizovaná rezidua $\times$ pořadí pozorování -- možná detekce řadové korelace mezi pozorováními.
	
	Obrázek....
\end{itemize}
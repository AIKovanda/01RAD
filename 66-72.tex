Z věty o~spektrálním rozkladu plyne existence $Q$ OG a~?? $\Lambda$ tak, že 
$$ \C=\Q^T \Lambda \Q=\Q^T \left( \begin{array}{cc}
I_m&\textbf{0}  \\
\textbf{0}& \textbf{0}
\end{array}
 \right)\Q,$$
 která má vlastní čísla $0$ a~$1$, protože se~jedná o~idenpotentní matici. Dále potom
 $$ \SSR=\textbf{e}^T\Q^T\left( \begin{array}{cc}
 I_m&\textbf{0}  \\
 \textbf{0}& \textbf{0}
 \end{array}
 \right)\underbrace{\Q\textbf{e}}_{\textbf{Z}\sim\NN(0,\sigma^2 I_m)}=\textbf{Z}^T\left( \begin{array}{cc}
 I_m&\textbf{0}  \\
 \textbf{0}& \textbf{0}
 \end{array}
 \right)\textbf{Z}=\sumin Z_i^2,$$
 kde $Z_i\sim\NN(0,\sigma^2)$ jsou nezávislé. Z~toho vyplývá, že 
 $$ \frac{Z_i}{\sigma}\sim\NN(0,1)\qquad\text{a}\qquad\frac{\SSR}{\sigma^2}\sim\chi^2(m).$$
 To znamená, že 
 $$ \frac{\frac{\SSR}{\sigma^2 m}}{\frac{(n-m-1)s_n^2}{\sigma^2}\frac{1}{n-m-1}}=\frac{\frac{\SSR}{m}}{s_m^2}=F\sim \FF(m,n-m-1),$$
 pokud ukážeme, že $\SSR$ a~$s_n^2$ jsou nezávislé. K~tomu ale stačí dokázat, že $\SSR$ je nezávislé na~reziduích $\he_i,~i\in\hat{n}$.
 $$ \SSR=\textbf{e}^T\textbf{H}\big( I_n-\frac{1}{n}\textbf{B} \big)\textbf{H}\textbf{e}=\textbf{e}^T\textbf{H}\underbrace{\big(I_n-\frac{1}{n}\textbf{B}\big)\big(I_n-\frac{1}{n}\textbf{B}\big)}_{=I_n-\frac{1}{n}\textbf{B}}\textbf{H}\textbf{e}=\frac{T}{\textbf{w}^T\textbf{w}},$$
 kde $\textbf{w}=\big(I_n-\frac{1}{n}\textbf{B}\big)\textbf{H}\textbf{e}\equiv \textbf{K}\textbf{e}$, $\widehat{\textbf{e}}=(I_n-\textbf{H})\textbf{e}\equiv \textbf{L}\textbf{e}$. Stačí tedy ukázat, že $\textbf{w}$ a~$\widehat{\textbf{e}}$ jsou nezávislé vektory. Víme, že 
 $$ \binom{\textbf{w}}{\widehat{\textbf{e}}}=\binom{\textbf{K}}{\textbf{L}} \textbf{e},$$
 tzn. má vícerozměrné normální rozdělení. Pokud je výraz $\textbf{K}\textbf{L}^T$ z~rovnice \ref{rovnicka} roven nule, pak jsou $\textbf{w}$ a~$\widehat{\textbf{e}}$ nezávislé.
  \begin{equation}\label{rovnicka}
  \Cov\binom{\textbf{w}}{\widehat{\textbf{e}}}=\binom{\textbf{K}}{\textbf{L}}\Cov~ \textbf{e}(\textbf{K}^T,\textbf{L}^T)=\sim^2 \left( \begin{array}{cc}
  \textbf{K}\textbf{K}^T&\textbf{K}\textbf{L}^T  \\
  \textbf{L}\textbf{K}^T & \textbf{L}\textbf{L}^T
  \end{array}
  \right)
  \end{equation} 
  Pro~$\textbf{K}\textbf{L}^T$ platí, že
  $$ \textbf{K}\textbf{L}^T=\big( I_n-\frac{1}{n}\textbf{B} \big)\underbrace{\textbf{H}(I_n-\textbf{H})}_{\textbf{H}-\textbf{H}^2=\textbf{0}}=0.$$
\end{proof}
 
 $\test{F>\FF_{1-\alpha}(m,n-m-1)}$
 
 \begin{remark}
 	Odvozeno pro~$e_i\sim\NN(0,\sigma^2)$, obecně se~používá, i~když to nevíme, pro~velké $n$ může být často zdůvodněno pomocí CLV.
 \end{remark}
\subsection*{Tabulka ANOVA}
$$ \begin{array}{l|cccc}
\text{Source}& \mathrm{df} & \mathrm{SS} & \mathrm{MS} & F \\\hline
\text{Regression} & m & \SSR & \MSR=\frac{\SSR}{m} & \frac{\MSR}{\MSE} \\
\text{Residual} & n-(m+1) & \SSE & \MSE=\frac{\SSE}{n-m-1}=s_n^2 &  \\
\text{Total} & n-1 & \SST &  &  \\\hline
&  & \RR^2 & \overline{\RR}^2 & 
\end{array}
 $$
 \subsection*{Koeficient (vícenásobná) determinace $\RR^2$}
 Podobně jako u~jednorozměrné regrese, lze F-test chápat jako test významnosti $\RR^2$, definovaného jako
 $$ \RR^2\equiv \frac{\SSR}{\SST}=1-\frac{\SSE}{\SST},$$
 protože 
 $$ F=\frac{\frac{\SSR}{m}}{\frac{\SSE}{n-m-1}}=\frac{n-m-1}{m}\left( \frac{\frac{\SSR}{\SST}}{\frac{\SSE}{\SST}} \right)=\frac{n-m-1}{m}\frac{\RR^2}{1-\RR^2},$$
 což je rostoucí funkce $\RR^2$ (opět $\RR^2\in[0,1]$).
 \begin{remark}
 	$\RR^2$ je možno zvětšovat přidáváním nových proměnných $x$, i~když jsou statisticky nevýznamné. (Pro $n$ LN proměnných $x$ a~$n$ pozorování dostaneme "perfect fit", tedy přeučení.) Vysvětlení:
 	$$ \RR^2=1-\frac{\SSE}{\SST},$$
 	kde $\SST$ je pevně dáno daty $y$, ale $\SSE$ může být snížena přidáním proměnných $x$. Minimalizujeme totiž $(\textbf{y}-\textbf{X}\boldsymbol{\beta})^T(\textbf{y}-\textbf{X}\wbeta)$ přes větší množinu $\boldsymbol{\beta}$. To znamená, že $\frac{\SSE}{\SST}$ je nerostoucí funkce počtu proměnných, a~tedy $\RR^2$ je neklesající funkce počtu proměnných. Z~tohoto důvodu se~někdy definuje \textbf{upravený koeficient determinace} (adjusted coefficient of determination)
 	$$ \overline{\RR}^2=\RR_{adj}^2=1-\frac{\frac{\SSE}{n-m-1}}{\frac{\SST}{n-1}}=1-\frac{n-1}{n-m-1}\frac{\SSE}{\SST}.$$
 	(S rostoucím $m$ klesá $\SSE$, ale i~$n-m-1$.)
 \end{remark}
\chapter{IS a~$t$-testy pro~parametry}
\begin{itemize}
	\item Pokud se~model ukáže jako významný, bude nás zajímat, které koeficienty přispívají.
	\item Lze použít IS a~TH stejně, jako u~jednorozměrné regrese.
	\item Výsledky jsou odvozeny pro~normální chyby.
	\item V~praxi se~používají i~pro~jiné typy chyb (za jistých předpokladů budou platit asymptoticky, lze je použít pro~velká $n$). 
\end{itemize}
Pro konstrukci použijeme dokázanou vlastnost
$$ T_j=\frac{\wbeta_j-\beta_j}{s_n\sqrt{v_j}}\sim t(n-m-1),\quad\text{kde}\quad v_j=(\textbf{X}^T\textbf{X})_{jj}^{-1}.$$
Standardním postupem získáme $100(1-\alpha)$\%. IS pro~$\beta_j$ ve~tvaru
$$\big( \wbeta_j-t_{1-\frac{\alpha}{2}}(n-m-1)s_n\sqrt{v_j},\wbeta_j+t_{1-\frac{\alpha}{2}}(n-m-1)s_n\sqrt{v_j} \big)$$
s jejich pomocí lze odvodit kritický obor pro~test
$$\hypothesis{\beta_j=b_j}{\beta_j\neq b_j}$$
ve tvaru
$$ \frac{|\wbeta_j-b_j|}{s_n\sqrt{v_j}}>t_{1-\frac{\alpha}{2}}(n-m-1).$$
Pro $b_j=0$ dostaneme test významnosti $\beta_j$, tzn. $H_0:~\beta_j=0$ zamítneme, pokud 
$$ \frac{|\wbeta_j|}{s_n\sqrt{v_j}}>t_{1-\frac{\alpha}{2}}(n-m-1).$$
\begin{remark}
	\begin{itemize}
		\item Pokud nejsou porušeny předpoklady modelu nebo není přítomna kolinearita, lze zvážit odstranění všech nevýznamných proměnných (dle t-testu).
		\item V~případě kolinearity, model může být významný (dle celkového F-testu), ale všechny nebo téměř všechny proměnné se~mohou jevit jako nevýznamné (dle t-testů).
		\item Naopak, pokud má model velký počet možných proměnných, některé proměnné se~mohou jevit významné, i~když jsou náhodným šumem.
		\item Při~použití t-testů je třeba být obezřetný.
	\end{itemize}
\end{remark}
\begin{example}
	5.26, str. 230 a~5.27, str. 231
\end{example}
\begin{remark}
	Statistiky $\FF,\RR^2$ a~$t$ jsou užitečné pro~rozkrytí efektů jednoduchých proměnných, nemohou být ale používány úplně automaticky.
\end{remark}
\subsection{Obecná lineární hypotéza}
F-test a~t-testy jsou speciálním případem \textbf{obecné lineární hypotézy}
$$ \hypothesiswide{\textbf{C}\boldsymbol{\beta}=\textbf{b}}{\textbf{C}\boldsymbol{\beta}\neq\textbf{b}},$$
kde $\textbf{C}\in\R^{r\times(m+1)}$ a~$h(\textbf{C})=r$, tzn. $r\leq m+1$. Rovnice $\textbf{C}\boldsymbol{\beta}=\textbf{b}$ reprezentuje $r$ lineárně nezávislých podmínek 
$$ \sum_{j=0}^m c_{ij}\beta_j=b_i,\qquad i=1,...,r.$$
\begin{remark}
	\begin{enumerate}[a)]
		\item Volba $\textbf{b}=(0,...,0)^T$ a~$\textbf{C}=\left(\begin{array}{c|cccc}
		0 & 1 & 0 & \dots & 0 \\\hline
		0 & 0 & 1 &  & \vdots \\
		\vdots& \vdots & \ddots & \ddots & 0 \\
		0 & 0 & \dots & 0 & 1
		\end{array}
		\right)_{m\times(m+1)}$ vede na~test 
		$$ H_0:~\textbf{C}\boldsymbol{\beta}=\textbf{0}\qquad\Leftrightarrow\qquad H_0:~\beta_1=\beta_2=...=\beta_m=0.$$
		\item Volba $\textbf{b}=\textbf{0}$ a~$\textbf{C}=(0,...,0,1,0,...,0)$ vede na~test 
		$$ H_0:~\beta_j=0.$$
		\item V~modelu $Y=\beta_0+\beta_1x_1+\beta_2x_2+\beta_3x_3+\beta_4x_4+e$ chceme testovat zároveň ,že $\beta_2=0$ a~$\beta_3=\beta_4$. To lze udělat volbou $\textbf{C}=\left(\begin{array}{ccccc}
		0 & 0 & 1 & 0 & 0 \\
		0 & 0 & 0 & 1 & -1
		\end{array}
		 \right),~\textbf{b}=(0,0)^T$.
	\end{enumerate}
\end{remark} 

Pro test $H_0$ naladíme 2 modely:\begin{description}
\item[plný model (full model)] bez~podmínek na~$\textbf{C}\boldsymbol{\beta}$,
\item[redukovaný model (reduced model)] za~předpokladu, že platí $H_0:~\textbf{C}\boldsymbol{\beta}=b$. 
\end{description}

Označme příslušné reziduální součty čtverců $\SSE_F$ a~$\SSE_R$ (bude platit $\SSE_F\leq\SSE_R$).

\begin{itemize}
	\item Pokud neplatí $H_0$, dá se~očekávat, že $\Delta\SSE=\SSE_R-\SSE_F$ bude významně větší, než náhodná chyba $\sigma^2$, $H_0$ tedy budeme zamítat, pokud $\frac{\Delta\SSE}{s_n^2}$ bude velké.
	\item Zobecnění F-testu, tj. za~platnosti $H_0$ ukázeme pro~normální chyby vztah 
	$$ F=\frac{\frac{\Delta\SSE}{r}}{s_n^2}\sim\FF(r,n-m-1).$$
\end{itemize}
\begin{example}
	Uvažujme F-test pro~$H_0:\beta_1=\beta_2=...=\beta_m=0$ v~plném modelu. Redukovaný model bude $Y_i=\beta_0+e_i,~i=1,...,n~\Rightarrow~\wbeta_0=\overline{\textbf{y}}$ a~$\SSE_R=\sumin(y_i-\ly)^2=\SST$, tedy 
	$$ \Delta \SSE=\SST-\SSE_P=\sumin(y_i-\ly)^2-\sumin(y_i-\hy_i)^2=\SSR$$
	 a~statistiku $F=\frac{\frac{\SSR}{m}}{s_n^2}=F_{overall}\sim\FF(m,n-m-1)$, jak jsme již ukázali.
\end{example}
\begin{theorem}
	Nechť v~modelu (**) platí, že $e_1,...,e_n$ jsou nezávislé a~$e_i\sim\NN(0,\sigma^2)$. Označme $\SSE_F$ reziuální s.č. plného modelu a~$\SSE_R$ reziduální s.č. modelu, kde platí $H_0:~\textbf{C}\boldsymbol{\beta}=\textbf{b}$. Potom, za platnosti $H_0$ je splněno
	$$ F=\frac{\frac{\Delta\SSE}{r}}{s_n^2}\sim\FF(r,n-m-1).$$
\end{theorem}